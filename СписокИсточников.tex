\addcontentsline{toc}{section}{СПИСОК ИСПОЛЬЗОВАННЫХ ИСТОЧНИКОВ}

\begin{thebibliography}{9}
    \bibitem{sis} Большаков И. А. Системы распознавания лиц – принцип работы и сферы применения / И. А. Большаков, С. А. Микаева. – Текст : непосредственный // Науносфера. – 2022. – № 9-2. – С. 95–98.
    \bibitem{abdulaev} Абдуллаев А. И. Распознавание лиц по изображению лица / А. И. Абдуллаев. – Текст : непосредственный // Мировая наука. – 2021. – № 4 (49). – С. 44–47.
    \bibitem{moscow} Кандидата наук Ермошина задержали как вора из-за ошибки распознавания лиц [Электронный ресурс] // Московский комсомолец. – 2021. – 19 октября. – URL: https://www.mk.ru/incident/2021/10/19/kandidata-nauk-ermoshina-zaderzhali-kak-vora-izza-oshibki-raspoznavaniya-lic.html? utm\_source=yxnews\&utm\_medium=desktop (дата обращения: 03.05.2025).
     \bibitem{four} Байкенов Б. С. Сравнительный анализ методов распознавания объектов / Б. С. Байкенов, Р. С. Тынчеров, А. Р. Фазылова. – Текст : непосредственный // Вестник Казахской академии транспорта и коммуникаций им. М. Тынышпаева. – 2019. – № 1 (108). – С. 185–191.
     \bibitem{history} Ветров С. В. История развития систем распознавания лиц / С. В. Ветров. – Текст : непосредственный // Наука и образование: актуальные исследования и разработки : материалы IV Всероссийской научно-практической конференции / отв. ред. А. В. Лесков. – Чита : Забайкальский государственный университет, 2021. – С. 23–29.
     \bibitem{six} Джеймс, Р. UML 2.0. Объектно-ориентированное моделирование и
     разработка / Р. Джеймс, Б. Майкл.– 2-е изд.– Санкт-Петербург : Питер, 2021.– 542 с.– ISBN 978-5-4461-9428-5.– Текст : непосредственный. 
     \bibitem{seven} Биэль, М. RESTful API Design / М. Биэль.– University Press, 2016.
     300 с.– ISBN 978-1-5147-3516-9.– Текст : непосредственный.
     \bibitem{eight} Аттуи, А. Real-Time and Multi-Agent Systems / А. Аттуи.– Springer
     Science \& Business Media, 2000.– 496 с.– ISBN 978-1-85233-252-5.– Текст :
     непосредственный.
     \bibitem{eleven} Буч, Г. Введение в UML от создателей языка / Г. Буч, И. Якобсон,
     Д. Рамбо.– Москва : ДМК Пресс, 2015.– 498 с.– ISBN 978-5-457-43379-3.
     Текст : непосредственный.
     \bibitem{twelve} Мандел, Т. Разработка пользовательского интерфейса / Т. Мандел.– ДМК Пресс, 2019.– 420 с.– ISBN 978-5-04-195060-6.– Текст : непосредственный.
     \bibitem{twelve} 5.	Мартин, Р. Чистая архитектура. Искусство разработки про-граммного обеспечения / Р. Мартин; пер. с англ. А. Кисилева. – Санкт-Петербург: Питер, 2018. – 351 с. – ISBN 978-5-4461-0772-8. – Текст: непо-средственный.
\end{thebibliography}
