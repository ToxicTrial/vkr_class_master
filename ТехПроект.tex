\section{Технический проект}
\subsection{Общая характеристика организации решения задачи}

Необходимо спроектировать и разработать систему, которая должна выполнять распознавание на основе изображений лиц и отпечатков пальцев с использованием карт пропуска.

Система представляет собой программу, позволяющую пользователю выполнять распознавание лиц и отпечатков пальцев и сканирование карты пропуска для авторизации. Интерфейс включает в себя окно, отображающее видеопоток с камеры и результаты распознавания лица или отпечатка пальца и карты пропуска. Программа автоматически определяет лицо, отпечаток пальца или карту пропуска в кадре, сопоставляет данные и отображает результаты выполнения программы на экране. 

\subsection{Обоснование выбора технологии проектирования}

Используемые для создания интеллектуальной системы языки программирования и технологии соответствуют современным практикам разработки, обеспечивают высокую производительность и отказоустойчивость системы.

\subsubsection{Описание используемых технологий и языков программирования}

В процессе разработки программы используется язык программирования Python, а также сторонние библиотеки OpenCV, PyTorch, NumPy, tkinter и Python Imaging Library.

\subsubsection{Язык программирования Python}

Язык программирования Python представляет собой высокоуровневый язык общего назначения, отличающийся лаконичным синтаксисом и высокой читаемостью кода. Python обеспечивает возможность реализации объектно-ориентированного, функционального и процедурного стилей программирования. Он обладает мощной стандартной библиотекой, которая охватывает множество областей применения — от работы с файлами и сетями до многопоточности и регулярных выражений. Python широко используется в научных вычислениях, благодаря таким библиотекам как NumPy, SciPy, Pandas и Matplotlib. В области машинного обучения и нейросетей Python является наилучшим языком программирования, благодаря таким инструментам как PyTorch, TensorFlow и Scikit-learn. Кроме того, язык поддерживается большим сообществом, что облегчает поиск решений и развитие проектов.

\subsubsection{Библиотека OpenCV}

OpenCV представляет собой мощную библиотеку компьютерного зрения и обработки изображений, предназначенную для выполнения широкого спектра задач, связанных с анализом изображений и видео. Она предоставляет высокоуровневые и низкоуровневые средства для работы с изображениями, включая функции для обработки, фильтрации, распозна-вания объектов и анализа движения. Основное преимущество OpenCV заключается в её высокой производительности, гибкости и поддержке широкого спектра алгоритмов компьютерного зрения. Благодаря интеграции с языком Python, OpenCV позволяет быстро разрабатывать приложения, связанные с обработкой изображений и видео, и является популярным инструментом как для учебных, так и для полноценных исследовательских и промышленных проектов.

\subsubsection{Библиотека PyTorch}

PyTorch — это современная библиотека машинного и глубокого обучения, предназначенная для создания и обучения нейронных сетей. Она предоставляет интуитивно понятные инструменты для работы с тензорами, автоматического дифференцирования и построения моделей. Одним из ключевых преимуществ PyTorch является использование динамической вычислительной графики, что делает разработку и отладку нейросетей более гибкой и наглядной. Библиотека поддерживает ускорение вычислений с помощью графических процессоров (GPU), что существенно повышает производительность при обучении моделей. Благодаря тесной интеграции с Python, а также множеству встроенных модулей и готовых архитектур, PyTorch широко применяется в задачах компьютерного зрения, обработки естественного языка, биометрии и других областях искусственного интеллекта.

\subsubsection{Библиотека NumPy}

NumPy представляет собой фундаментальную библиотеку для численных вычислений в Python, обеспечивающую поддержку многомерных массивов и высокоуровневых математических операций над ними. Благодаря высокоэффективным операциям с массивами и встроенным линейным алгебраическим методам, NumPy позволяет обрабатывать изображения больших размеров с минимальными затратами ресурсов, что делает её незаменимой в задачах, связанных с сжатием, декомпрессией и преобразованием графических данных.

\subsubsection{Библиотека tkinter}

Для построения графического интерфейса в проекте выбрана библиотека tkinter, которая является частью стандартной поставки Python и представляет собой обёртку над библиотекой Tcl/Tk. Этот выбор объясняется рядом технологических и практических преимуществ:
\begin{enumerate}
	\item Отсутствие необходимости в установке сторонних зависимостей делает tkinter особенно удобным для использования в проектах с упрощённым процессом развёртывания. Пользователь может сразу запустить приложение без дополнительных установок, что критично в условиях учебной и демонстрационной среды.
	\item Простота проектирования интерфейса позволяет легко создавать окна, формы, кнопки, меню, вкладки и другие элементы без необходимости изучения сложных графических фреймворков. Это способствует быстрому созданию прототипов и конечных версий интерфейса.
	\item Полная кроссплатформенность интерфейса гарантирует его одинаковое отображение и поведение на всех операционных системах, что избавляет разработчика от необходимости писать отдельные реализации под разные платформы.
	\item Наличие компонентов высокого уровня, таких как текстовые поля, выпадающие списки, кнопки, панели и таблицы, даёт возможность создать функциональный и интуитивно понятный интерфейс для работы с данными.
	\item Возможность динамического обновления интерфейса обеспечивает удобную визуализацию изменений.
\end{enumerate}

tkinter полностью удовлетворяет требованиям к простому, лёгкому в использовании, но функциональному графическому интерфейсу, особенно в условиях ограниченного времени и ресурсов.	

\subsubsection{Python Imaging Library}

Python Imaging Library, сокращенно PIL — это библиотека для работы с растровыми изображениями, обеспечивающая поддержку различных форматов, таких как JPEG, PNG, BMP и других. PIL предоставляет широкий набор функций для загрузки, сохранения, обработки и преобразования изображений, что упрощает различные взаимодействия с файлами изображений, такие как сохранение и загрузка изображений, поэтому она является удобным инструментом для предварительной и финальной обработки графических данных.



\subsection{Архитектура программной системы}

Точкой входа в программу является модуль main.py, в котором создаётся и запускается экземпляр класса App, реализующего основной графический интерфейс приложения. Архитектура программы ориентирована на модульность и разделение ответственности, что облегчает поддержку, тестирование и возможное расширение проекта.

Архитектура приложения состоит из следующих ключевых модулей:
\begin{enumerate}
	\item programm.py — стартовый модуль, в котором производится инициализация пользовательского интерфейса. Здесь создаётся объект MainWindow, который затем запускается методом mainloop(). Содержит в себе код основных классов приложения(App и LoadingScreen).
	
	\item func.py — содержит функции детекции лиц, извлечения эмбеддингов, сравнения лиц и работы с моделями для распознавания. Модуль отвечает за большинство вычислений в системе.
	
	\item model.py — реализует классы FingerprintEncoder и SiameseNetwork, предназначенные для обработки изображений отпечатков пальцев. Класс FingerprintEncoder — это нейронная сеть, основанная на архитектуре ResNet-18, модифицированная для извлечения эмбеддингов изображений отпечатков пальцев. Класс SiameseNetwork — это основная модель, использующая подход сиамской нейронной сети, где сравниваются два изображения. Эта сеть обучается таким образом, чтобы оценить, являются ли два изображения схожими или нет.
	
	\item dataset.py — создает пользовательскую выборку для обучения модели с использованием пар изображений отпечатков пальцев. Модуль создает как положительные, так и отрицательные пары изображений, которые используются для обучения сиамской сети.
	
	\item train.py — представляет собой процесс обучения модели для сравнения отпечатков пальцев с использованием сиамской нейронной сети и последующего сохранения обученной модели.
	
	\item fp.py — содержит реализацию обработки векторных представлений в классе Fingerprints. Класс Fingerprints используется для сравнения двух изображений отпечатков пальцев и оценки их схожести. Система использует модель Siamese Network, обученную для нахождения эмбеддингов (векторных представлений) изображений отпечатков пальцев, и затем вычисляет схожесть между этими эмбеддингами с использованием косинусного сходства.
	
	\item save\_face.py — захват изображений лиц с камеры и их сохранения в определенную директорию для дальнейшей тренировки модели распознавания лиц.
	
	\item embedding\_create.py — используется для извлечения векторных представлений лиц из изображений и сохранения их для последующего распознавания личности по лицу.
	
\end{enumerate}

Компоненты программы взаимодействуют следующим образом:
\begin{enumerate}
	\item Интерфейс (класс App из файла programm.py) получает данные пользователя, например, запрос на распознавание лица или отпечатка пальца, и передаёт их на обработку соответствующим функциям.
	
	\item Модуль func.py обрабатывает эти команды: выполняет детекцию лиц, извлекает эмбеддинги с помощью нейросетевых моделей, сравнивает их с ранее сохранёнными представлениями и возвращает результат.
	
	\item Результаты распознавания отображаются в пользовательском интерфейсе. При этом данные могут быть изменены или обновлены в зависимости от действий пользователя.
	
	\item Модули save\_face.py и embedding\_create.py используются для подготовки базы данных лиц: save\_face.py захватывает изображения с камеры, а embedding\_create.py извлекает и сохраняет их эмбеддинги.
	
	\item Для распознавания отпечатков пальцев используется модель, реализованная в файле model.py и обученная с помощью модуля train.py. Сравнение векторных представлений отпечатков выполняется в модуле fp.py на основе косинусного расстояния.
	
	\item Все изображения и векторные представления хранятся в структуре каталогов database/, что позволяет системе работать автономно, без внешней базы данных.
\end{enumerate}

Между компонентами поддерживается минимальная необходимая связь: графический интерфейс не содержит логики обработки выражений, а логика ядра независима от визуального представления. Это упрощает поддержку, тестирование и адаптацию системы для различных платформ.

Архитектура гибкая, расширяемая и легко масштабируемая. Возможность добавления новых типов команд или подключения внешних источников данных или API без необходимости переписывания основной логики делает систему адаптируемой к разнообразным задачам.
